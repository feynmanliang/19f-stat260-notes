\section{10/8/2019}

\subsection{Resilience Beyond Mean Estimation}

\begin{itemize}
  \item General framework: still $D = \TV$
  \item Will generalize
    \begin{itemize}
      \item Modulus of continuity
      \item Resilience
      \item Analogue of $\cG_{\Cov}(\sigma)$
        \begin{itemize}
          \item Moment estimation
          \item Linear regression
        \end{itemize}
    \end{itemize}
\end{itemize}

Recall our general framework for robust statistics
(\cref{fig:robust-statistics-framework}, reproduced below):
\begin{figure}[H]
    \centerline{
        \xymatrix{
            \txt{train \\ $\tilde{p}$} \ar[d] \ar@{<->}[r]^-{D(\tilde{p}, p^*) \leq \eps} \ar[dr]
            & \txt{test\\$p^* \in \cG$} \ar[dr] \\
            \txt{$X_1, \ldots, X_n$\\samples} \ar[r] &
            \txt{$\hat\theta(\tilde{p})$\\estimator} \ar[r] &
            \txt{$L(p^*, \hat\theta)$\\loss}
        }
    }
\end{figure}

For now, assume $n = \infty$ so we neglect finite sample issues
and our estimator $\hat\theta(\tilde{p})$ directly uses the population
distribution $\tilde{p}$.

\begin{example}
  $D = \TV$ and $L(p, \theta) = \|\theta - \mu(p)\|_2$ (mean-estimation).

  For second-moment estimation, we can consider
  $L(p, \underbrace{S}_{\in \RR^{d \times d}}) = \|S - \ex_p[X X^\top] \|$.
  This only measures the top eigenvector, so sometimes a more natural loss is
  \begin{align}
    \|I - \Sigma^{-1} \Cov_p[X]\|_F
  \end{align}
  where the Frobenius norm now weights all eigenvalues equally.

  In linear regression, the excess squared loss
  $L(p, \theta) = \ex_{(x,y) \sim p}[(y - \braket{\theta, x})^2 - (y - \braket{\theta^*(p), x})^2]$.
\end{example}

\subsubsection{MDF modulus bound}

Recall for a family $\cG$, the modulus of continuity
\begin{align}
  \fm(\cG, 2\eps) &= \sup_{\substack{p, q \in \cG \\ \TV(p,q) \leq 2 \eps)}}
  \underbrace{\|\mu(p) - \mu(q)\|_2}_{L(p, \theta)}
\end{align}
We can generalize it to other losses as
\begin{align}
  \fm(\cG, 2\eps) &= \sup_{\substack{p, q \in \cG \\ \TV(p,q) \leq 2 \eps)}}
  L(p, \theta^*(q))
\end{align}
There is a suitable generalization of \myref{prop:mdf-modulus-error-bound}
\begin{proposition}[Minimum distance functional $\leq \fm$]
  \begin{align}
    \hat\theta(\tilde{p}) = \theta^*(q)~\text{where}~q = \argmin_{q \in \cG} \TV(\tilde{p}, q)
  \end{align}
  By assumption, $q \in \cG$ and $\TV(\tilde{p}, q) \leq \eps$.

  Figure 10.8.1

  \begin{align}
    \TV(p^*, q) &\leq 2 \eps \\
    L(p^*, \theta^*(q)) \leq \fm(\cG, 2 \eps)
  \end{align}
  $(p^*, q \in \cG)$
\end{proposition}

\subsubsection{Resilience}

Recall $p$ is $(\rho, \eps)$-resilient if
\begin{align}
  \|\mu(p) - \mu(r)\|_2 \leq \rho~\text{whenever}~r \leq \frac{p}{1-\eps}
\end{align}
$\cG_{\TV}(\rho, \eps)$ denotes the set of all $(\rho, \eps)$-resilient distributions.

\begin{lemma}
  If $\TV(p, q) \leq \eps$, there is a \emph{midpoint} $r$ such that
  $r \leq \frac{p}{1-\eps}$, $r \leq \frac{q}{1-\eps}$.
\end{lemma}

Figure 10.8.2: For resilient distributions, $\fm(\cG_\TV, \eps) \leq 2 \rho$.

We used symmetry properties of mean estimation quite extensively in the above
argument, so we need to extract these properties in order to generalize
resilience:

\begin{definition}[Generalized resilience]\label{def:resilience-general}
  $p$ is \emph{$(\rho_1, \rho_2, \eps)$-resilient (for general $L$)} if
  \begin{description}
    \item[Downwards condition] $L(r, \theta^*(p)) \leq \rho_1$ for all $r \leq \frac{p}{1-\eps}$.
      So doing well on $p$ means we also do well on $\eps$-deletions
    \item[Upwards condition] If $L(r, \theta) \leq \rho_1$ for any $r \leq \frac{p}{1-\eps}$
      and $\theta$, then $L(p, \theta) \leq \rho_2$. So doing well
      on all $\eps$-deletions means that we also do well on $p$.
  \end{description}
  We have two conditions to generalize beyond mean estimation, because these
  losses may not have symmetry.
\end{definition}

\begin{example}[Mean estimation]
  $L(p, \theta) = \|\mu(p) - \theta\|_2$, $\theta^*(p) = \mu(p)$.

  The downward condition:
  \begin{align}
    L(r, \theta^*(p)) &= \|\mu(r) - \mu(p)\|_2 \\
    \|\mu(r) - \mu(p)\|_2 &\leq \rho \qquad \forall r \leq \frac{p}{1-\eps}
  \end{align}

  The upward condition:
  \begin{align}
    \left.
    \begin{array}{c}
      \text{if }\|\mu(r) - \theta\|_2 \leq \rho_1 \\
      \text{then } \|\mu(r) - \theta\|_2 \leq \rho_2
    \end{array}
    \right\}
    \|\mu(r) - \mu(p)\|_2 \leq \rho_2 - \rho_1
  \end{align}
  % $\|\mu(p) - \theta\|_2 \leq \|\mu(r) - \theta\|_2 + \|\mu(r) - \mu(p)\|_2 \leq \rho_2$.

  So generalized resilience here is saying the means are close, consistent
  with the previous definition (\cref{def:resilience}).
\end{example}

$(\rho,\eps)$-resilience $\iff$ $(\rho, 2\rho, \eps)$-resilience.

\begin{proposition}
  $\cG_{\downarrow}(\rho_1, \eps) = \{\text{ all $p$ satisfying $\downarrow$}\}$

  $\cG_{\uparrow}(\rho_1, \rho_2, \eps) = \{\text{ all $p$ satisfying $\uparrow$}\}$

  % $\cG_{\TV}(\rho_1, \rho_2, \eps) = \cG_{\downarrow}(\rho_1, \eps) \cap \cG_{\uparrow}(\rho_1, \rho_2, \eps)$.

  Then $\fm(\cG_{\TV}(\rho_1, \rho_2, \eps), \eps) \leq \rho_2$.
\end{proposition}

\begin{proof}
  Need to show $p, q \in \cG_{\TV}$, $\TV(p, q) \leq \eps$, we have
  $L(p, \theta^*(q)) \leq \rho_2$.

  Figure 10.8.3: $\cG_\downarrow$ was used to move $q \to r$
  and $\cG_\uparrow$ to move $r \to p$.
\end{proof}

\begin{remark}
  This theory only holds for $D = \TV$, otherwise we are not guaranteed $\exists r$.
\end{remark}

\begin{example}[Second moment estimation]
  $L(p, S) = \|S - \ex_p[ X X^\top]\|$.

  $\cG_\downarrow(\rho, \eps) \implies \|\ex_r[x x^\top] - \ex_p[x x^\top]\| \leq \rho_1$
  whenever $r \leq \frac{p}{1-\eps}$.
  This is just saying that $X X^\top$ is (old) $(\rho_1, \eps)$-resilient under
  operator norm.

  $\cG_\uparrow(\rho, \eps) \implies \|\ex_r[X X^\top] - S\| \leq \rho_1$,
  $\rho_2 = 2 \rho_1$, $\implies \|\ex_p[x x^\top] - S\| \leq \rho_2$.

  We will show $X X^\top$ is $(2 \sigma \sqrt{\eps}, \eps)$-resilient in operator
  norm $\Leftarrow$ $\Var[\braket{X X^\top, Z}] \leq \sigma^2$ whenever $\|Z\|_* \leq 1$.

  For $\|\cdot\|$ the operator norm, the dual norm is the nuclear norm
  \begin{align}
    \|Z\|_* = \sum_i \sigma_i(Z)
  \end{align}
  where $\sigma_i(Z)$ are the singular values of $Z$.
  For $\Z\|_* \leq 1$, the extreme points are $\pm v v^\top$
  with $\|v\|_2 \leq 1$.

  \begin{align}
    \Var[\braket{X X^\top, v v^\top}]
    &= \Var[\braket{X, v}^2]
    \leq \ex[\lvert \braket{X, v}^2 \rvert^2]
    = \ex[\braket{x, v}^4]
    \underbrace{\leq}_{\text{WTS}} \sigma^2
  \end{align}
  If we have $\ex[\braket{x, v}^4]^{1/4} \leq \tau$,
  then $\sigma = \tau^2$ and we get $(2 \tau^2 \sqrt{\eps}, \eps)$-resilience.

  This works for any $k$ (bounded $2k$th moments), so we get
  $(2 \sigma \eps^{1 - 1/k}, \eps)$-resilience (mean)
  and $(2 \sigma^2 \eps^{1 - 2/k}, \eps)$-resilience (second moment).
\end{example}


\begin{proposition}[Linear regression]
  Suppose $(X, Y) \sim p$, $L$ is excess squared loss,
  $Z = Y - \braket{\theta^*(p), X}$.
  If
  \begin{enumerate}
    \item $\ex[\braket{x, v}^4] \leq \kappa \ex[\braket{x, v}^2]^2$ for all $v$
    \item $\ex[X Z^2 X^\top] \preceq \sigma^2 \ex[X X^\top]$
  \end{enumerate}
  and $\eps \leq 1/2$, $\eps(\kappa - 1) \leq 1 / 16$, then $p$
  is $(\rho, 5\rho, \eps)$-resilient with $\rho = 3 \sigma \sqrt{\eps}$.
\end{proposition}

\begin{proof}
  Some general observations:
  \begin{itemize}
    \item $L(p, \theta) = \ex_p[(y - \braket{\theta, x})^2 - (y - \braket{\theta^*(p), x})^2]
      = (\theta - \theta^*(p))^\top S_p (\theta - \theta^*(p))$
      where $S_p = \ex_p[X X^\top]$.
    \item $S_r \approx S_p$
    \item $\theta^*(r) \approx \theta^*(p)$
  \end{itemize}

  \begin{lemma}
    If $\eps (\kappa - 1) \leq \frac{1}{16}$, then
    \begin{align}
      \frac{1}{2} S_p \preceq S_r \preceq \frac{3}{2} S_p
    \end{align}
    if $r \leq \frac{p}{1-\eps}$
  \end{lemma}
  \begin{proof}
    For all $r$
    \begin{align}
      v^\top S_p v - v^\top S_r v
      &= \lvert \ex_p[\braket{x,v}^2] - \ex_r[\braket{x,v}^2]\rvert
      \leq 2 \sqrt{\eps \Var_p[\braket{x,v}^2]}
    \end{align}
    Furthermore
    \begin{align}
      \Var_p[\braket{x,v}^2]
      &= \underbrace{\ex_p[\braket{x,v}^4]}_{\leq \kappa \ex_p[\braket{x,v}^2]^2}
      - \ex_p[\braket{x,v}^2]^2
    \end{align}
    Hence
    \begin{align}
      \lvert \ex_p[\braket{x,v}^2] - \ex_r[\braket{x,v}^2]\rvert
      &\leq 2 \sqrt{\eps (\kappa - 1)} \ex_p[\braket{x,v}^2]
      \leq \frac{1}{2} \ex_p[\braket{x,v}^2]
    \end{align}
    Hence $\ex_r[\braket{x,v}^2] \in (1/2 \ex_p[\braket{x,v}^2], 3/2 \ex_p[\braket{x,v}^2])$.
  \end{proof}

  Note $\theta^*(r) - \theta^*(p) = S_r^{-1} \ex_r[X Z]$.
  Take for granted $\| S_r^{-1/2} \ex_r[X Z]\|_2 = \cO(\sigma \sqrt{\eps})$ (will prove next time).

  For now we analyze the upwards an downwards conditions. Starting with $\cG_{\downarrow}$
  \begin{align}
    (\theta^*(p) - \theta^*(r))^\top S_r (\theta^*(p) - \theta^*(r)) 
    &\leq \rho \\
    (S_r^{-1} \ex_r[X Z])^\top S_r (S_r^{-1} \ex[X Z])
    &=  \ex_r[X Z]^\top S_r^{-1} \ex_r[X Z] \\
    &= \|S_r^{-1/2} \ex_r[X Z]\|_2
  \end{align}
  For $\cG_{\uparrow}$, want to show
  \begin{align}
    (\theta - \theta^*(r))^\top S_r (\theta - \theta^*(r)) 
    \leq \rho_1
    &\implies (\theta - \theta^*(p))^\top S_p (\theta - \theta^*(p))
  \end{align}
  To do this, will expand $(\theta - \theta^*(p) + \theta^*(p) - \theta^*(r))$
  and replace $S_r$ with $S_p$, then apply triangle inequality.
\end{proof}
